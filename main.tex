\documentclass[11pt]{article}

\usepackage[utf8]{inputenc}
\usepackage[T1]{fontenc}
\usepackage{graphicx}
\usepackage{color,soul}
\usepackage[usenames,dvipsnames,table,svgnames]{xcolor}
\usepackage{array}
\usepackage{listings}
\usepackage{amsmath}
\usepackage[absolute,showboxes]{textpos}
\usepackage[locale=DE]{siunitx}
\sisetup{detect-all} 
\usepackage{amssymb}
\usepackage{float}
\usepackage{enumitem}
\usepackage[ngerman]{datetime}
\usepackage{textcomp}
\usepackage{gensymb}
\usepackage{makecell}
\usepackage{geometry}
\usepackage{textgreek}
\usepackage{tabularx}
\usepackage{fancyhdr}
\usepackage{titling}
\usepackage{lastpage}
\usepackage{framed}
\usepackage{stackengine}
\usepackage{tocloft}
\usepackage{chngcntr}
\usepackage{rotating}
\usepackage{capt-of}
\usepackage{wrapfig}
\usepackage{fancyref}
\usepackage[strict]{changepage}
\usepackage[hypcap=false]{caption}
\usepackage{subcaption}
\usepackage[ngerman]{babel}
\usepackage{titlesec}
\usepackage[hidelinks]{hyperref}
\newcommand\xrowht[2][0]{\addstackgap[.5\dimexpr#2\relax]{\vphantom{#1}}}
\setlength{\FrameSep}{3pt}
\makeatletter

\usepackage[sfdefault]{noto}

\definecolor{hka-green}{RGB}{33,154,52}

%%% Author Infos %%%
\author{Max Mustermann}                                                         % S-S-S-Slim Shady
\newcommand{\authorID}{12345}                                                   % Matrikel-Nr

%%% Document Infos %%%
\title{Irgendein Titel, das wahrscheinlich viel zu lang und kompliziert ist}    % Was schreib ich nochmal?
\newcommand{\subject}{Informationstechnik}                                      % Vertiefungsrichtung
\newcommand{\keywords}{Informationstechnik, Bachelorthesis}                     % PDF keywords
\newcommand{\referent}{Prof. Dr.-Ing. Prof X}                                   % Korrektester, den man gefunden hat
\newcommand{\korreferent}{Prof. Dr.-Ing. Prof Y}                                % Zweitkorrektester, aber eigentlich egal
\newcommand{\company}{Irgendeine Firma GmbH \& Co. KG}                          % Wo du dabei Kohle machst
\newcommand{\supervisor}{Der Betreuer}                                          % Wem du knechtest
\newcommand{\timeframe}{14.07.2022 - 13.11.2022}                                % Zeit der Anmeldung -- Zeit der Abgabe
\newcommand{\semester}{Sommersemester 2022}

%%% ToC %%%
\renewcommand{\cfttoctitlefont}{\huge\bfseries\color{hka-green}}                
\renewcommand{\cftloftitlefont}{\huge\bfseries\color{hka-green}}

\renewcommand{\cftsecfont}{\large\bfseries\color{hka-green}}
\renewcommand{\cftsubsecfont}{\normalsize}
\renewcommand{\cftsubsubsecfont}{\normalsize}

\renewcommand{\cftsecpagefont}{\large\bfseries\color{hka-green}}
\renewcommand{\cftsubsecpagefont}{\normalsize}
\renewcommand{\cftsubsubsecpagefont}{\normalsize}

\renewcommand{\cftsubsecdotsep}{3}
\renewcommand{\cftsubsubsecdotsep}{3}
\renewcommand{\cftpnumalign}{c}

\renewcommand{\cftsecnumwidth}{1cm}
\renewcommand{\cftsubsecnumwidth}{1cm}
\renewcommand{\cftsubsubsecnumwidth}{1.4cm}

\renewcommand{\cftsecaftersnum}{.}
\renewcommand{\cftsubsecaftersnum}{.}
\renewcommand{\cftsubsubsecaftersnum}{.}

\setlength{\cftbeforesecskip}{15pt}
\setlength{\cftbeforesubsecskip}{5pt}
\setlength{\cftbeforesubsubsecskip}{2pt}

\newcommand\mydot[1]{\scalebox{#1}{.}}
\renewcommand\cftdot{\mydot{0.5}}

%%% Titles %%%
\titleformat*{\section}{\huge\bfseries\color{hka-green}}
\titleformat*{\subsection}{\Large\bfseries}
\titleformat*{\subsubsection}{\large\bfseries}
\titlelabel{\thetitle.\quad}

%%% Page Style %%%
\setlength{\parindent}{0pt}
\setlength{\parskip}{0.1in}
\geometry{margin=2cm,footskip=1.2cm} % Setup for pagestyle

%%% Header & Footer %%%
\pagestyle{fancy}   % Sooo fancy
\fancyhf{}
\renewcommand{\headrulewidth}{0pt}

%%% Captions %%%
\DeclareCaptionFormat{hka-caption}{
    \color{hka-green}\textbf{#1#2}\textit{\small #3}
}
\captionsetup{format=hka-caption}

%%% Counters %%%
\counterwithin{figure}{section}     % Kann kopiert werden für equations usw

%% Bibliography
\usepackage[style=ieee]{biblatex}  
\addbibresource{References.bib}     % Wer traut sich da Wikipedia reinzupacken?
\usepackage{fvextra}
\usepackage[german=quotes]{csquotes}

%% Footnotes
\usepackage[bottom]{footmisc}
\renewcommand{\footnoterule}{}

%% Acronyms & LoA
\usepackage{acro} %
%% List
\newlist{acronyms}{description}{100}
\setlist[acronyms]{labelwidth=3cm,leftmargin=1cm,itemsep=3pt,itemindent=2cm,font=\normalfont\bfseries}
%\DeclareAcroListStyle{custom}{list}{list = acronyms}
\acsetup{
    make-links = true,
    format =,
    list/display=all,
    list/heading=section*,
    list/name={Abkürzungsverzeichnis\\},
}

%% Acronyms
\DeclareAcronym{fpga}{
  short=FPGA,
  long=Field Programable Gate Array,
}    % Alles Nur Aus Liebe
% Für die ganz extremen unter euch kann man auch ein Abkürzungsverzeichnis mit dem Package packen. Ich lasse euch den Spaß :*


% % Minted   --- Use with shell-escape flag
% % Sieht zwar besser aus als listings aber braucht so unglaublich lange zum kompilieren. Dir überlassen. 
% \usepackage{ifthen}
% \usepackage{currfile-abspath}
% \getabspath{\jobname.log}
% \ifthenelse{\equal{\theabsdir}{\thepwd}}{}{\PassOptionsToPackage{outputdir=\theabsdir}{minted}}
% \usepackage{minted}
%%%%%%%%%%%%%%%%%%%%%%%%%%%%%%%%%%%%%%%%%%%%%%%%%%%%%%%%%%%%%%%%%%%%%%%%%%%%%%%%%%%%%%%%%%%%%%%%%%%%%%%%%%%%%%%%%%%%%%%%%%%%%%%%%%%%%%%

\begin{document}

% HKA Title page
% Langweiliges Zeug hier. Wenn du deine Bachelorarbeit in der Hochschule machst kannst du eigentlich die Zeilen Arbeitsplatz und Betreuer löschen im tabular
%%% PDF Properties %%%
\hypersetup{                                                                        
    pdftitle    = {\thetitle},                                   
    pdfsubject  = {\subject},    
    pdfauthor   = {\theauthor},                                       
    pdfkeywords = {\keywords} ,                        
    pdfcreator  = {pdflatex},           
    pdfproducer = {LaTeX with hyperref}
}

%%% TITLE PAGE %%%
\begin{titlepage}
    \newgeometry{left=3cm,right=3cm,top=1cm,bottom=2cm,footskip=1.2cm}
    \begin{minipage}[t]{13cm}\raggedright
        \includegraphics[width=5cm]{Template_HKA/hka_address.png}
        \vspace{4cm}

        \Large
        Studiengang Elektro- und Informationstechnik –    % Fraktion Nerds
        Vertiefung \subject
    \end{minipage}
    \hspace{1.5cm}
    \begin{minipage}{3cm}\raggedright
        \includegraphics[width=\textwidth]{Template_HKA/hka_bachelor.png}
    \end{minipage}
    
    \vspace{1cm}
    \color{hka-green}
    \Huge\textbf{Bachelor-Thesis}  % In anderen Worte das Ende von 5 Jahren Krampf (wenn es weniger sind tuts mir leid für dich)
                                   % Champagne ! 

    \vspace{1cm}
    \color{black}\huge
    \textbf{\thetitle}

    \vspace{2cm}
    \Large
    von\\
    \color{hka-green} \textbf{\theauthor}  % Booooom
    \vfill
    \color{black}\normalsize{}
    \begin{tabular}{ll} % das ding da
        Matrikel-Nummer:            & \quad \authorID                           \\[1ex]
        Referent:                   & \quad \referent                           \\[1ex]
        Korreferent:                & \quad \korreferent                        \\[4ex]
        % VV Die beiden mein ich VV
        Arbeitsplatz:               & \quad \company                            \\[1ex]
        Betreuer am Arbeitsplatz:   & \quad \supervisor                         \\[4ex]
        Zeitraum:                   & \quad \timeframe                          \\[1ex]
    \end{tabular}
\end{titlepage}
\newpage  % Sieht geil aus ich weiß, danke
\newgeometry{margin=2cm,footskip=1.2cm}

% ToC
\tableofcontents
\thispagestyle{empty}  
\newpage

% Footer
% Agefackter scheiss hier unten, dafür da, dass die letzte subsection oder section im footer angezeigt wird
\renewcommand{\subsectionmark}[1]{\markboth{#1}{}}
\renewcommand{\sectionmark}[1]{\markboth{#1}{}}
\renewcommand{\footrulewidth}{0.1pt}
\rfoot{\thepage \hspace{1pt} von \pageref{LastPage} }
\cfoot{\leftmark}  % Das kannste wegmachen wenn du nicht magst

% Body
% Hier kommt eigentlich dein Zeug. Include macht nützlicherweise einen clearpage nach der section, kann aber auch mit input importiert werden
\section{Section 1}
Test \cite*{MSI:Radars}
\subsection{Subsection 1}
\subsubsection{Subsubsection 1}
\subsubsection{Subsubsection 2}
\subsection{Subsection 2}
\include{Sections/section_2.tex}
\include{Sections/section_3.tex}

% Bibliography   
\phantomsection
\cfoot{}
\addcontentsline{toc}{section}{Literaturverzeichnis}
\printbibliography  % Da ist der Beweis, dass man keinen Müll labert. Zweitrangig.
\newpage

% LoF
\phantomsection
\addcontentsline{toc}{section}{Abbildungsverzeichnis}
\listoffigures   % Habe nie verstanden warum dass dabei sein muss. Wer schaut sich das bitte an ?!
\thispagestyle{fancy}  % SOOOO FANCYYY
\newpage

\end{document}